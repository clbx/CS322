\documentclass[12pt,letterpaper]{article}
\usepackage{fullpage}
\usepackage[top=2cm, bottom=4.5cm, left=2.5cm, right=2.5cm]{geometry}
\usepackage{amsmath,amsthm,amsfonts,amssymb,amscd}
\usepackage{lastpage}
\usepackage{enumerate}
\usepackage{fancyhdr}
\usepackage{mathrsfs}
\usepackage{xcolor}
\usepackage{graphicx}
\usepackage{listings}
\usepackage{hyperref}

\hypersetup{%
  colorlinks=true,
  linkcolor=blue,
  linkbordercolor={0 0 1}
}
 
\renewcommand\lstlistingname{Algorithm}
\renewcommand\lstlistlistingname{Algorithms}
\def\lstlistingautorefname{Alg.}

\lstdefinestyle{C++}{
    language        = C++,
    frame           = lines, 
    basicstyle      = \footnotesize,
    keywordstyle    = \color{blue},
    stringstyle     = \color{green},
    commentstyle    = \color{red}\ttfamily
}

\setlength{\parindent}{0.0in}
\setlength{\parskip}{0.05in}

% Edit these as appropriate
\newcommand\course{CS322 Algorithms}
\newcommand\hwnumber{1}                  % <-- homework number
\newcommand\NetIDb{Clay Buxton}           % <-- NetID of person #1

\pagestyle{fancyplain}
\headheight 35pt
\lhead{\NetIDa}
\lhead{\NetIDa\\\NetIDb}                 % <-- Comment this line out for problem sets (make sure you are person #1)
\chead{\textbf{\Large Homework \hwnumber}}
\rhead{\course \\ \today}
\lfoot{}
\cfoot{}
\rfoot{\small\thepage}
\headsep 1.5em

\begin{document}

\section*{Problem 1}
\textbf{Prove gcd(m,n) \textbar{} gcd(n,m\%n)}\\

Let c = gcd(n,m\%n) for all integers n,m where n \textgreater m\\
Therefore c\textbar{}n and c\textbar{}m\%n\\
If c\textbar{}m\%n then c\textbar{}km+n for some integer k\\
if c\textbar{}km+n then c\textbar{}gcd(m,n) by definition of gcd

\section*{Problem 2}
\textbf{Implement the Sieve algorithm}
\textit{Also see primes.cpp}

\lstset{caption={Sieve Algorithm}}
    \lstset{label={lst:alg1}}
     \begin{lstlisting}[style = C++]
        std::vector<int> primes::sieve(int n){
            //Initlize List
            std::vector<int> primes;
            for(int i = 0; i < n; i++){
                primes.push_back(i);
            }
            for (int i = 2; i <= n; i++){
                if(!std::count(primes.begin(),primes.end(),i){
                    continue;
                }
                for (int j = i + i; j <= n; j += i){
                    primes.erase(std::remove(primes.begin(), primes.end(), j), primes.end());
                }
            }
            return primes;
        }          
    \end{lstlisting}
\newpage

\section*{Problem 3}



\begin{enumerate}
    \item
    \textbf{Implement the Extended Euclidean Algorithm}
    \textit{Also see euclidean.cpp}
    \lstset{caption={Extended Euclidean Algorithm}}
    \lstset{label={lst:alg2}}
     \begin{lstlisting}[style = C++]
    int euclidean::gcdExtended(int a, int b, int *x, int *y){  
        if (a == 0){  
            *x = 0;  
            *y = 1;  
            return b;  
        }  
        int x1, y1; 
        int gcd = gcdExtended(b%a, a, &x1, &y1);  
    
        *x = y1 - (b/a) * x1;  
        *y = x1;  
  
        return gcd;
    }
    \end{lstlisting}
    \item
    \textbf{Find the integer solution to the Diophantine problem}
    \textit{Also see euclidean.cpp}
    
        

    \lstset{caption={Diophantine Equation}}
        \lstset{label={lst:alg2}}
         \begin{lstlisting}[style = C++]
    void euclidean::diophantine(int a, int b, int c, int *x, int *y) {
        int gcd = gcdExtended(a,b,x,y);
        if(c%gcd){
            std::cout << "No Solution" << std::endl;
        }
        int d = c / gcd;
        *x *= d;
        *y *= d;
        return;
    }
        \end{lstlisting}


\end{enumerate}
\section*{Problem 4}
\begin{enumerate}
    \item 14,25,47,60,81,98
    \item No, if given two numbers of the same value the algorithm will not work.
    \item No, it uses 2 arrays the "Count" and "A" arrays, and then needs another array to output the values to.
\end{enumerate}



\end{document}