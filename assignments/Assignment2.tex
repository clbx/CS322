\documentclass[12pt,letterpaper]{article}
\usepackage{fullpage}
\usepackage[top=2cm, bottom=4.5cm, left=2.5cm, right=2.5cm]{geometry}
\usepackage{amsmath,amsthm,amsfonts,amssymb,amscd}
\usepackage{lastpage}
\usepackage{enumerate}
\usepackage{fancyhdr}
\usepackage{mathrsfs}
\usepackage{xcolor}
\usepackage{graphicx}
\usepackage{listings}
\usepackage{hyperref}
\usepackage{amssymb}

\hypersetup{%
  colorlinks=true,
  linkcolor=blue,
  linkbordercolor={0 0 1}
}
 
\renewcommand\lstlistingname{Algorithm}
\renewcommand\lstlistlistingname{Algorithms}
\def\lstlistingautorefname{Alg.}

\lstdefinestyle{C++}{
    language        = C++,
    frame           = lines, 
    basicstyle      = \footnotesize,
    keywordstyle    = \color{blue},
    stringstyle     = \color{green},
    commentstyle    = \color{red}\ttfamily
}

\setlength{\parindent}{0.0in}
\setlength{\parskip}{0.05in}

% Edit these as appropriate
\newcommand\course{CS322 Algorithms}
\newcommand\hwnumber{2}                  % <-- homework number
\newcommand\NetIDb{Clay Buxton}           % <-- NetID of person #1

\pagestyle{fancyplain}
\headheight 35pt
\lhead{\NetIDa}
\lhead{\NetIDa\\\NetIDb}                 % <-- Comment this line out for problem sets (make sure you are person #1)
\chead{\textbf{\Large Homework \hwnumber}}
\rhead{\course \\ \today}
\lfoot{}
\cfoot{}
\rfoot{\small\thepage}
\headsep 1.5em

\begin{document}

\section*{Problem 1}
\begin{tabular}{p{0.5\linewidth}p{0.5\linewidth}}
\textbf{a.)} \textbf{Same}. $n(n+1) = n^2+n$. $n^2+n$ and $2000n^2$ have the same order of growth. & \textbf{b.)} \textbf{Lower}. $n^2$ is a lower order of growth than $n^3$ regardless of coefficient\\
\textbf{c.)} \textbf{Same}. All logarithms have the same order of growth. & \textbf{d.)}\textbf{Higher}. $log_2^2(n) = (log_2(n))^2$ and $log_2(n^2) = 2(log_2(n))$\\
\textbf{e.)} \textbf{Same}. $2^{n-1} = \frac{1}{2}(2^n)$ and $\frac{1}{2}(2^n)$ has the same order of magnitude as $2^n$ & \textbf{f.)} \textbf{Lower}. $(n-1)!$ is one order of growth lower then $n!$\\
\end{tabular}

\section*{Problem 2}
\begin{description}
    \item [a.)]$C_{worst}(n) \in \theta(n)$.\\ 
    $C_{worst}(n) = n$ Since there are n number of elements in the array. If the key is either the last element or not in the array, the order of execution is n.\\\\
    For $C_{worst}(n) \in O(n)$.  $\exists c$ such that $cn \leq n$. 
    Let $c = 1$
    $cn \leq n$ \xrightarrow{} $1n \leq n$ \xrightarrow{} $n \leq n$\\\\
    \therefore $C_{worst} \in O(n)$\\\\
    For $C_{worst}(n) \in \Omega(n)$.  $\exists c$ such that $cn \geq n$. 
    Let $c = 1$
    $cn \geq n$ \xrightarrow{} $1n \geq n$ \xrightarrow{} $n \geq n$\\\\
    \therefore $C_{worst} \in \Omega(n)$\\\\
    Since $C_{worst} \in O(n)$ and $C_{worst} \in \Omega(n), C_{worse} \in \Theta(n)$\\
    
    \item [b.)] $C_{best}(n) \in \theta(1)$.\\
    $C_{best}(n) = 1$ Since the best case is the first element of the array.\\\\
    By definition $C_{best} \in \Omega(1)$ and $C_{best} \in O(1)$\\\\
    And if $C_{best} \in \Omega(1)$ and $C_{best} \in O(1)$ then $C_{best} \in \theta(1)$
    
    
    \item [c.)]$C_{avg}(n) \in \theta(n)$\\
    $C_{avg}(n) = \tfrac{p(n+1))}{2}+n(1-p)$\\\\
    For $C_{avg}(n) \in O(n), c(\tfrac{p(n+1))}{2}+n(1-p)) \leq \tfrac{p(n+1))}{2}+n(1-p)$\\\\
    Let $0 < c < 1$ so $c(\tfrac{p(n+1))}{2}+n(1-p)) \leq \tfrac{p(n+1))}{2}+n(1-p)$\\\\
    $\therefore C_{avg} \in O(n)$\\\\
    For $C_{avg}(n) \in \Omega(n), c(\tfrac{p(n+1))}{2}+n(1-p)) \geq \tfrac{p(n+1))}{2}+n(1-p)$\\\\
    Let $c > 1$ so $c(\tfrac{p(n+1))}{2}+n(1-p)) \geq \tfrac{p(n+1))}{2}+n(1-p)$\\\\
    $\therefore C_{avg} \in \Omega(n)$
    
    And if $C_{avg} \in O(n)$ and $C_{avg} \in \Omega(n)$ then $C_{avg} in \Theta(n)$ 
    
\end{description}

\section*{Problem 3}

$5lg(n+100)^{100}$ \xrightarrow{} $ln^2n$ \xrightarrow{} $\sqrt[3]{n}$ \xrightarrow{} $.001n^4+3n^3+1$ \xrightarrow{} $3^n$ \xrightarrow{} $2^{2n}$ \xrightarrow{} $(n-2)!$

\section*{Problem 4}

\begin{description}
    \item[a.)] If $f(n) = n$ and $g(n) = n + sin(n)$ then $g(f(n)) = f(n) + sin(f(n))$ which means that $g(n) \leq f(n)$ \therefore $g(n) \in O(f(n))$ 
    \item[b.)] If $f(n) = n$ and $g(n) = n|sin(n)|$ then $g(f(n)) = f(n)|sin(f(n))|$ which means that $g(n) \leq f(n)$ \therefore $g(n) \in O(f(n))$ 
    
\end{description}



\end{document}
